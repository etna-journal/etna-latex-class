%
% Copyright (C) 2013-2014 by André Gaul, Nico Schlömer
%
% This file may be distributed and/or modified under the
% conditions of the LaTeX Project Public License, either
% version 1.2 of this license or (at your option) any later
% version. The latest version of this license is in:
%
%    http://www.latex-project.org/lppl.txt
%
% and version 1.2 or later is part of all distributions of
% LaTeX version 1999/12/01 or later.
%
\documentclass{etna}

\title{Dummy title}
\author{John Doe\footnotemark[1]
\and
Jean Dupont\footnotemark[1]
\and
Max Mustermann\footnotemark[2]
}

\newtheorem{remark}[theorem]{Remark}
\newtheorem{example}[theorem]{Example}

\usepackage{booktabs}
\usepackage{lipsum}

\begin{document}

\maketitle

\footnotetext[1]{Department of Mathematics, University of Foobar, Foo
  Road 14, Germany (\texttt{\{doe,dupont\}@foobar.de}).}
\footnotetext[2]{Institut f\"ur Mathematik, Universit\"at Berlin,
  Hauptstr. 1, Germany (\texttt{mustermann@berlin.de}).}


\begin{abstract}
  \lipsum[1]
\end{abstract}

\begin{keywords}
   dummy keyword, foo bar, foo, bar
\end{keywords}

\begin{AMS}
  00A01, 02B03, 04C05
\end{AMS}

\section{Introduction}
\lipsum[2]

\begin{theorem}[Pythagoras]
For a right triangle with legs $a$ and $b$ and hypotenuse $c$,
$a^2+b^2=c^2$.
\end{theorem}
\begin{theorem}[Brouwer]
Any continuous function $G:B^n\to B^n$ has a fixed point, where
\[
  B^n=\{x \in R^n:x_1^2+\dots+x_n^2\le 1\}
 \]
is the unit $n$-ball.
\end{theorem}
\begin{theorem}[Fisher]
Let $A$ be a sum of squares of $n$ independent normal standardized variates
$X_i$, and suppose $A=B+C$ where $B$ is a quadratic form in the $x_i$,
distributed as $\chi$-squared with $h$ degrees of freedom. Then $C$ is
distributed as $\chi^2$ with $n-h$ degrees of freedom and is independent of
$B$.
\end{theorem}

\lipsum[3]

\begin{theorem}[Cauchy]\label{th:1}
  Tic tic tic,
  \begin{equation}\label{eq:1}
    f^{(n)}(a) = \frac{n!}{2\pi i} \oint_\gamma \frac{f(z)}{(z-a)^{n+1}}\, dz.
  \end{equation}
\end{theorem}
\begin{proof}
See~\cite{MR2882785}. $\qquad$
\end{proof}

Reference to equation~(\ref{eq:1}), theorem~\ref{th:1}.

Another equation:
\begin{equation}
  R = \frac{\displaystyle{\sum_{i=1}^n (x_i-\bar{x})(y_i-
\bar{y})}}{\displaystyle{\left[
\sum_{i=1}^n(x_i-\bar{x})^2
\sum_{i=1}^n(y_i-\bar{y})^2\right]^{1/2}}}
\end{equation}

\begin{table}
\centering
\caption{A dummy table.}
\begin{tabular}{llr}
\toprule
\multicolumn{2}{c}{Item} \\
\cmidrule(r){1-2}
Animal    & Description & Price (\$) \\
\midrule
Gnat      & per gram    & 13.65      \\
          & each        & 0.01       \\
Gnu       & stuffed     & 92.50      \\
Emu       & stuffed     & 33.33      \\
Armadillo & frozen      & 8.99       \\
\bottomrule
\end{tabular}
\end{table}

\begin{remark}\upshape
  \lipsum[4]
\end{remark}
\begin{example}
  \lipsum[5]
\end{example}
\subsection{Corollaries}
\begin{corollary}
  \lipsum[6]
\end{corollary}
\begin{lemma}
  \lipsum[27]
\end{lemma}

\section{Conclusions}
\lipsum[10-12]

\begin{figure}
  \caption{A dummy figure.}
\end{figure}

\appendix

\section{Miscellaneous lemmas}
\lipsum[13-15]

\bibliographystyle{siam}
\bibliography{example.bib}

\end{document}
