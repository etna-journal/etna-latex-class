\documentclass[10pt]{article}

\def\qed{~\vbox{\hrule\hbox{\vrule height1.3ex\hskip0.8ex\vrule}\hrule}}


\begin{document}

\begin{center}{\bf\Large
Guidelines for ETNA manuscripts\footnote{Version of May 13, 2009.}%\\[0.25ex]
%J\"org Liesen, draft of \today
}\end{center}

\bigskip

\section{General formatting guidelines}

A manuscript for ETNA must be written in English. It may be
in color provided it is equally readable when displayed in
black and white.\\[0.5ex]
%
Any manuscript submitted to ETNA must be developed in {\LaTeX}
using the style files in the SIAM {\LaTeX} package {\tt siamltex}.
This means that the manuscript must follow
the general layout defined by the SIAM {\LaTeX} style.\\[0.5ex]
%
There are several versions of the SIAM {\LaTeX} package. For ETNA
manuscripts the version available from the ETNA web site should be
used\footnote{{\tt http://etna.math.kent.edu/LaTeX}}. This package
contains the following 5 files:\\[0.5ex]
%
\noindent {\tt siam.bst} (version of January 24, 1988)\\
{\tt siam10.clo} (Version 1.0; October 1, 1995)\\
{\tt siamltex.cls} (Version 1.1; August 1, 1995)\\
{\tt siamltex.sty} (Version 1.0; December 1, 1995)\\
{\tt subeqn.clo} (last modified June 8, 1989)\\[0.5ex]
%
\noindent Two further files are available from the {\LaTeX} directory
at the ETNA web site: The file {\tt docultex.tex} contains a documentation
for SIAM macros for use with {\LaTeX} 2e. The file {\tt lexample.tex}
contains examples.\\[0.5ex]
%
Compiling the two files  {\tt docultex.tex} and {\tt lexample.tex}
results in two 8-page manuscripts that explain all major issues in
the context of the SIAM {\LaTeX} style and give numerous examples.
%In the following section ETNA specific rules will be explained.



\section{ETNA specifics}
All manuscripts must follow the following ETNA specific rules
(see Figure~\ref{fig:preamble} at the end of this document for an
example of the initial part of a paper according to the
SIAM {\LaTeX} style and with ETNA specific commands):

\begin{itemize}
\item ETNA papers are compiled using the sequence {\tt latex} $\rightarrow$
{\tt dvips} $\rightarrow$ {\tt ps2pdf}, i.e., the dvi file is
translated into PostScript, and the PostScript file is translated
into PDF. To make this sequence work, it is generally recommended
that {\em all figure files are in PostScript format (ps or eps).}
Figure files in other formats, e.g. PDF, jpg, etc., are usually not
acceptable. Before submitting a manuscript to ETNA, authors should
make sure that the manuscript can be compiled in the above mentioned
sequence.
%
\item The SIAM {\LaTeX} style used by ETNA supports the
use of the package {\tt epsfig} for including PostScript figures. See
the {\tt epsfig} documentation for details on the use of this style option.
%
\item Equation numbers are placed on the left of the equation. This is
achieved by adding {\tt leqno} to the ``documentclass'' options.
%
\item Typesetting of ETNA papers is based on the {\tt times} package.
The command
\begin{verbatim}     \usepackage{times}\end{verbatim}
has to be loaded in the document's preamble.
%
% TODO: adapt numbering description, e.g. 'consecutive numbering'.
%
\item Examples, remarks, algorithms, etc., should be numbered in the
same way and format as theorems, namely as ``Example X.Y'', where X is the section
number and Y is the subsection number (if applicable). To generate the theorem-like
environments ``Example'', ``Remark'', and ``Algorithm'' you should add the commands
\begin{verbatim}
     \newtheorem{remark}[theorem]{Remark}
     \newtheorem{example}[theorem]{Example}
     \newtheorem{algorithm}[theorem]{Algorithm}
\end{verbatim}
to the preamble of your tex-file.
%
\item The text of examples and remarks should be typeset in roman
(not in italics as in definitions, theorems, lemmas, and corollaries).
This can be achieved as follows:
\begin{verbatim}
     \begin{remark}
     {\rm Text of remark.}
     \end{remark}
\end{verbatim}
%
\item  At the end of a proof, there should be a gap between the last
word and the ``endproof'' symbol (Halmos box). This is achieved as follows:
%
\begin{verbatim}     \begin{proof} Text of the proof.  $\qquad$ \end{proof}\end{verbatim}
%
If the proof ends with a displayed equation you should use the following:
%
\begin{verbatim}
     {\em Proof}. Text of the proof.
     $$a^2+b^2=c^2. \qquad\endproof$$
\end{verbatim}
%
You can also generate a command for the ``endproof'' symbol by adding
%
\begin{verbatim}
     \def\qed{~\vbox{\hrule\hbox
        {\vrule height1.3ex\hskip0.8ex\vrule}\hrule}}
\end{verbatim}
%
to your tex-file. Typing {\tt \char`\\qed} in a formula then produces
the symbol $\qed\,$.
%
\item ETNA uses the {\tt hyperref} package that
produces hypertext links in the document. The package must be loaded
in the document's preamble as follows:
%
\begin{verbatim}
     \usepackage
        [dvips,
        letterpaper=true,
        colorlinks=true,
        linkcolor=red,
        filecolor=green,
        citecolor=red,
        pdfpagemode=None]
        {hyperref}
\end{verbatim}

\item All references in the manuscript should be ``clickable'', which
means that all numbers of sections, definitions, theorems, equations, etc.,
should be labeled. {\em Equations that are not referenced should not be numbered.}
%
\item Tables should have as few lines as possible. For example,
instead of\\

\hspace{2cm}\begin{tabular}{|c|c|c|}\hline
$C1$ & $C2$ & $C3$ \\ \hline
$a_1$ & $a_2$ & $a_3$ \\ \hline
$b_1$ & $b_2$ & $b_3$ \\ \hline
\end{tabular}\\

you should use\\

\hspace{2cm}\begin{tabular}{c|c|c}
$C1$ & $C2$ & $C3$ \\ \hline
$a_1$ & $a_2$ & $a_3$ \\
$b_1$ & $b_2$ & $b_3$
\end{tabular}~~.\\

\item Table captions should be {\em above} the table (unlike figure captions which
should be below the figure). The text of all captions (table or figure)
should end with a period ``.''.

\item Appendices should be at the end of the paper, immediately before the
bibliography.

\item Further style-related comments:
\begin{itemize}
\item The symbols used for real, complex numbers, etc.,
are {\tt {\char`\\mathbb\{R\}}}, {\tt {\char`\\mathbb\{C\}}}, etc.
%
\item For matrices, square brackets should be used \\
(either use
{\tt \char`\\left[ ... \char`\\right]} or the {\tt \char`\\bmatrix}
command).
%
\item Sections should be referred to by ``Section~X.Y'' and not ``\S~X.Y''.
%
\item The abbreviations ``et al.'', ``i.e.'', and ``e.g.'', in the text
should always be in roman font.
In English, there is always a comma after ``i.e." and ``e.g.".
%
A parenthetical remark in the middle of the sentence
(e.g., this one) is in parenthesis, while at the end
of the sentence follows a semicolon; e.g., this one.
%Insertions in the middle of the sentence should be in parenthesis, but at
%the end of the sentence they should be separated by a semicolon. Example:
%``The middle of the sentence (i.e., here), and the end of the sentence;
%i.e., here.''
%
\end{itemize}
\end{itemize}


\section{References}
To facilitate the editing process, authors are especially urged to
carefully prepare the references of their manuscripts using the
correct abbreviations of names of serials. The correct abbreviations
can be found, for example, using {\em MathSciNet} at {\tt
http://ams.rice.edu/mathscinet/searchjournals}.
%A PDF file of the
%correct abbreviations can also be found at
%{\tt http://www.ams.org/msnhtml/serials.pdf}.
%
{\em Manuscripts may be returned to authors if the manuscripts and the
references are not properly prepared.} The preferred way for formatting
the references is to use BIBTeX with the SIAM bibliography style:

\begin{center}\begin{verbatim}
         \bibliographystyle{siam}
         \bibliography{YourBibFile}
\end{verbatim}\end{center}
%
If you cannot use BIBTeX, here are a few examples of correct
references for ETNA:

\begin{itemize}
\item Examples for paper citations:

\begin{verbatim}
\bibitem{ArbGol88}
\textsc{P. Arbenz and G.~H. Golub},
\emph{On the spectral decomposition of {H}ermitian matrices
modified by low rank perturbations with applications},
SIAM J. Matrix Anal. Appl., 9 (1988), pp.~40--58.

\bibitem{ArbGol95}
\sameauthor,
\emph{Matrix shapes invariant under the symmetric {QR} algorithm},
Numer. Linear Algebra Appl., 2 (1995), pp.~87--93.
\end{verbatim}
%
Note: In the second (and further) paper(s) by the same author(s) the
name(s) should be replaced by the command {\tt \char`\\sameauthor}, which
is defined by the SIAM LaTeX style. Instead of {\tt \char`\\sameauthor} you may use
``\mbox{\tt \char`\\leavevmode\char`\\vrule height 2pt depth -1.6pt width 23pt}''.

\item Examples for papers that are submitted or to appear:

\begin{verbatim}
\bibitem{Cay58}
\textsc{A. Cayley},
\emph{Generalising Hamilton's theorem to higher order matrices},
Math. Ann., submitted, 1858.

\bibitem{Gau10}
\textsc{C.~F. Gauss},
\emph{Another proof of the fundamental theorem of algebra},
J. Reine Angew. Math., to appear, 2010.
\end{verbatim}

Note: Giving the journal name is not required, and the publication year
should only be added if known.

\item Example for book citations:

\begin{verbatim}
\bibitem{FoxPar68}
\textsc{L. Fox and I.~B. Parker},
\emph{Chebyshev Polynomials in Numerical Analysis},
Oxford University Press, Oxford, UK, 1968.
\end{verbatim}

Note: All major words in book titles have to be capitalized.


\item Example for citations from collections:

\begin{verbatim}
\bibitem{Mor90}
\textsc{J.~J. Mor\'e},
\emph{A collection of nonlinear model problems},
in Computational Solutions of Nonlinear Systems of Equations,
E.~L. Allgower and K. Georg, eds., Lectures in Appl. Math., 26,
Amer. Math. Soc., Providence, RI, 1990, pp.~723--762.
\end{verbatim}

\end{itemize}

\noindent Theses or reports: It is the author's responsibility to provide
complete details such as editors, publisher, city of publication, page
numbers, and department and institution.\\

\noindent Further instructions concerning the references:

\begin{itemize}
\item In the text references should be cited using the command
{\tt \char`\\cite$\{\mbox{\tt keylist}\}$},
where {\tt keylist} contains a list of keys (i.e., the names of your bibitems),
separated by commas. Example:
{\tt \char`\\cite$\{\mbox{\tt ArbGol88,ArbGol95}\}$}.
%
\item The list of references should be ordered alphabetically, and
citations should be sorted according to numbers, that is [3,5,12],
instead of [12,3,5]. It is fine when you order citations by publication
dates, but this should be consistent throughout the paper.
%
\item Only references cited in the text should be included in the bibliography.
%
\item Citations in the abstract should be avoided. If this is not possible,
then the authors' names and the publication details should be given. For example,

``Golub and Kahan [SIAM J. Numer. Anal., 2 (1965), pp.~205--224] show that ...''.

\item When a reference is available online, the URL should be added to the citation, e.g.,
%
\begin{verbatim}
\url{http://etna.math.kent.edu/vol.26.2007/pp453-473.dir}
\end{verbatim}

\item Some journals have ``paper numbers'' instead of conventional page numbers.
An example is the Journal of Fluids Engineering. Papers in such journals should
be cited using the paper number and the number of pages; e.g.,
J. Fluids Eng., 130 (2008), 051202 (10 pages).


\end{itemize}

%\newpage

\section{Submission}
To submit a manuscript, send a ready-to-print PostScript or PDF file containing
the whole manuscript by email to
%
\begin{verbatim}     etna@math.kent.edu\end{verbatim}
%
or on a CD to ETNA, Department of Mathematical Sciences, Kent State University,
Kent, Ohio 44242, USA. The e-mail address of the author to contact for revisions
and corrections must be included on the disk in an ASCII-file called ``address.txt".



\newpage
\begin{figure}
\begin{verbatim}
\documentclass[final,leqno]{siamltex}

\usepackage{times}
\usepackage[dvips,letterpaper=true,colorlinks=true,
        linkcolor=red,filecolor=green,citecolor=red,
        pdfpagemode=None]{hyperref}

\title{Paper's title\thanks{Received ... Accepted for publication ...
       Recommended by ... Work supported by ...}}

\author{First author's name\footnotemark[2]
        \and Second author's name\footnotemark[3]}

% Note: For authors from the same institution there should be one
%       common footnote and the email addresses in the footnote text
%       should be written as ({\tt {authorA,authorB}@uofi.edu}).

\begin{document}

\maketitle

\renewcommand{\thefootnote}{\fnsymbol{footnote}}

\footnotetext[2]{First author's address.}
\footnotetext[3]{Second author's address.}

\begin{abstract}
Paper's abstract.
\end{abstract}

\begin{keywords}
Paper's key words, non-capitalized except for names and the first word.
\end{keywords}

\begin{AMS}
AMS subject classifications.
\end{AMS}

\pagestyle{myheadings}\thispagestyle{plain}
\markboth{F. AUTHOR AND S. AUTHOR}{SHORT TITLE}

\section{First section} This is typically the paper's Introduction.
....
\end{verbatim}
\caption{Example for the initial part of a paper prepared according to the
SIAM {\LaTeX} style and with some ETNA specific commands.}\label{fig:preamble}
\end{figure}



\end{document}
