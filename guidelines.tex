\documentclass[parskip]{scrartcl}

\def\qed{~\vbox{\hrule\hbox{\vrule height1.3ex\hskip0.8ex\vrule}\hrule}}

\title{Guidelines for ETNA manuscripts\footnote{Version of \today.}}
%J\"org Liesen, draft of \today
\date{}

%\usepackage[T1]{fontenc}
%\usepackage{lmodern}
%\usepackage{microtype}
\usepackage{booktabs}
\usepackage
[
colorlinks=true,
linkcolor=red,
filecolor=green,
citecolor=red
]
{hyperref}

\begin{document}

\maketitle

\section{General formatting guidelines}

A manuscript for ETNA must be written in English. It may be in color provided
it is equally readable when displayed in black and white.

Any manuscript submitted to ETNA must be developed in {\LaTeX} using the style
files in the SIAM {\LaTeX} package \texttt{siamltex}.  This means that the
manuscript must follow the general layout defined by the SIAM {\LaTeX} style.

There are several versions of the SIAM {\LaTeX} package. For ETNA manuscripts
the version available from the ETNA web site should be
used\footnote{\url{http://etna.math.kent.edu/LaTeX}}. This package contains the
following 5 files:

\texttt{siam.bst} (version of January 24, 1988)\\
\texttt{siam10.clo} (Version 1.0; October 1, 1995)\\
\texttt{siamltex.cls} (Version 1.1; August 1, 1995)\\
\texttt{siamltex.sty} (Version 1.0; December 1, 1995)\\
\texttt{subeqn.clo} (last modified June 8, 1989)

Two further files are available from the {\LaTeX} directory at the ETNA web
site: The file \texttt{docultex.tex} contains a documentation for SIAM macros
for use with {\LaTeX}2e. The file \texttt{lexample.tex} contains examples.

Compiling the two files \texttt{docultex.tex} and \texttt{lexample.tex} results
in two 8-page manuscripts that explain all major issues in the context of the
SIAM {\LaTeX} style and give numerous examples.
%In the following section ETNA specific rules will be explained.

\section{ETNA specifics}
All manuscripts must follow the following ETNA specific rules (see
Figure~\ref{fig:preamble} at the end of this document for an example of the
initial part of a paper according to the SIAM {\LaTeX} style and with ETNA
specific commands):

\begin{itemize}
\item ETNA papers are compiled using \texttt{lualatex}.
\item Figures should be submitted as vector graphics and not as bitmaps whenever
possible, e.g., with the native \LaTeX{} packages Ti\textit{k}Z or Pgfplots, or
  as PDF or SVG files.

\item For authors from the same institution there should be one
common footnote and the email addresses in the footnote text
should be written as (\texttt{{authorA,authorB}@uofi.edu}).

%
% TODO: adapt numbering description, e.g. 'consecutive numbering'.
%
\item Examples, remarks, algorithms, etc., should be numbered in the same way
  and format as theorems, namely as ``Example X.Y'', where X is the section
  number and Y is the subsection number (if applicable). To generate the
  theorem-like environments ``Example'', ``Remark'', and ``Algorithm'' you
  should add the commands
\begin{verbatim}
     \newtheorem{remark}[theorem]{Remark}
     \newtheorem{example}[theorem]{Example}
     \newtheorem{algorithm}[theorem]{Algorithm}
\end{verbatim}
to the preamble of your \TeX-file.
%
\item The text of examples and remarks should be typeset in roman (not in
  italics as in definitions, theorems, lemmas, and corollaries).  This can be
  achieved as follows:
\begin{verbatim}
     \begin{remark}\upshape
       Text of remark.
     \end{remark}
\end{verbatim}
%
\item  At the end of a proof, there should be a gap between the last word and
  the ``endproof'' symbol (Halmos box). This is achieved as follows:
%
\begin{verbatim}
    \begin{proof} Text of the proof.  $\qquad$ \end{proof}
\end{verbatim}
%
If the proof ends with a displayed equation you should use the following:
\begin{verbatim}
     \emph{Proof.} Text of the proof.
     \[
       a^2+b^2=c^2. \qquad\endproof
     \]
\end{verbatim}

\item Equations that are not referenced should not be numbered.
%
\item Tables should be typeset with the booktabs packages and should have as few lines as possible and no vertical lines.
For example, instead of
\begin{center}
  \begin{tabular}{|c|c|c|}\hline
  $C1$ & $C2$ & $C3$ \\ \hline
  $a_1$ & $a_2$ & $a_3$ \\ \hline
  $b_1$ & $b_2$ & $b_3$ \\ \hline
  \end{tabular}
\end{center}
the form
\begin{center}
  \begin{tabular}{ccc}\toprule
  $C1$ & $C2$ & $C3$ \\ \midrule
  $a_1$ & $a_2$ & $a_3$ \\
  $b_1$ & $b_2$ & $b_3$\\
  \bottomrule
  \end{tabular}
\end{center}
should be used.
\item Table captions should be \emph{above} the table (unlike figure captions
  which should be below the figure). The text of all captions (table or figure)
  should end with a period~``.''.

\item Appendices should be at the end of the paper, immediately before the
bibliography.

\item Further style-related comments:
\begin{itemize}
\item The symbols used for real, complex numbers, etc.,
are \texttt{{\char`\\mathbb\{R\}}}, \texttt{{\char`\\mathbb\{C\}}}, etc.
%
\item For matrices, square brackets should be used \\
(either use
\texttt{\char`\\left[ \dots \char`\\right]} or \texttt{\char`\\bmatrix}).
%
\item Sections should be referred to by ``Section~X.Y'' and not ``\S~X.Y''.
%
\item The abbreviations ``et al.'', ``i.e.'', and ``e.g.'', in the text
should always be in roman font.
In English, there is always a comma after ``i.e.'' and ``e.g.''.
%
A parenthetical remark in the middle of the sentence (e.g., this one) is in
parenthesis, while at the end of the sentence follows a semicolon; e.g., this
one.
%Insertions in the middle of the sentence should be in parenthesis, but at
%the end of the sentence they should be separated by a semicolon. Example:
%``The middle of the sentence (i.e., here), and the end of the sentence;
%i.e., here.''
%
\end{itemize}
\end{itemize}


\section{References}
To facilitate the editing process, authors are especially urged to carefully
prepare the references of their manuscripts.
It is the author's responsibility to provide complete details such as editors,
publisher, city of publication, page numbers, department and institution, and
correct abbreviations of
names of serials. All of this information can be found, for example, using
\emph{MathSciNet}\footnote{\url{http://www.ams.org/mathscinet/}}.

%A PDF file of the
%correct abbreviations can also be found at
%\texttt{http://www.ams.org/msnhtml/serials.pdf}.
%
\emph{Manuscripts may be returned to authors if the manuscripts and the
references are not properly prepared.} The preferred way for formatting the
references is to use BibTeX with the SIAM bibliography style:
\begin{verbatim}
         \bibliographystyle{siam}
         \bibliography{YourBibFile}
\end{verbatim}

Further instructions concerning the references:

\begin{itemize}
\item In the text references should be cited using the command
  \texttt{\char`\\cite$\{\mbox{\texttt{keylist}}\}$},
where \texttt{keylist} contains a list of keys (i.e., the names of your bibitems),
separated by commas. Example:
\texttt{\char`\\cite$\{\mbox{\texttt{ArbGol88,ArbGol95}}\}$}.
%
\item Citations should be either sorted according to numbers, that is [3,5,12]
instead of [12,3,5], or by publication dates. The order should be consistent
throughout the paper.
%
\item Only references cited in the text should be included in the bibliography.
%
\item Citations in the abstract should be avoided. If this is not possible,
then the authors' names and the publication details should be given. For example,

\begin{quote}
  Golub and Kahan [SIAM J. Numer. Anal., 2 (1965), pp.~205--224] show that
  [\dots].
\end{quote}

%\item If a reference is available online, the URL should be added to the
%  citation, e.g.,
%%
%\begin{verbatim}
%\url{http://etna.math.kent.edu/vol.26.2007/pp453-473.dir}
%\end{verbatim}

\item Some journals have ``paper numbers'' instead of conventional page
  numbers.  An example is the Journal of Fluids Engineering. Papers in such
  journals should be cited using the paper number and the number of pages,
  e.g., the bibtex \texttt{pages} field should be of the form \texttt{051202 (10 pages)}.


\end{itemize}

\section{Submission}
To submit a manuscript, send a PDF file containing
the whole manuscript by email to
\href{mailto:etna@math.kent.edu}{\nolinkurl{etna@math.kent.edu}}.

\end{document}
