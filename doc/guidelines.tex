\documentclass[parskip=half]{scrartcl}

\title{Guidelines for ETNA manuscripts}
% Original document by J\"rg Liesen,
% adapted by Andr\'e Gaul and Nico Schl\"omer

\date{\today}

\usepackage[T1]{fontenc}
\usepackage{lmodern}
\usepackage{microtype}

\usepackage{booktabs}
\usepackage[
colorlinks=true,
linkcolor=red,
filecolor=green,
citecolor=red
]
{hyperref}

\usepackage{minted}

\begin{document}

\maketitle

%\begin{abstract}
%This document describes the preparation of a manuscript for submission to
%Electronic Transactions on Numerical Analysis (ETNA).
%\end{abstract}

A manuscript for ETNA must be written in English. It may be in color provided
it is equally readable when displayed in black and white.

Any manuscript submitted to ETNA must be developed in \LaTeX{} using the ETNA
style files. The manuscript must follow the general guidelines described in
this document.
ETNA papers are compiled using \texttt{lualatex}, so before submitting, make
sure that your document compiles cleanly.

\section{Installation}

The ETNA style files are provided at
\begin{quote}
\url{http://etna.math.kent.edu/LaTeX}.
\end{quote}
For installation, the package needs to be downloaded an unpacked in a directory
where \LaTeX{} can find it.

\subsection*{Quick start}
The ETNA package contains an example document, \texttt{example.tex} which can
be used as a template for preparing a manuscript.

\section{Usage}
All manuscripts submitted to ETNA must use the ETNA style, i.e.,
\verb|\documentclass{etna}|, and adhere to the following ETNA specific rules.

\begin{itemize}
\item Figures should be submitted as vector graphics and not as bitmaps
  whenever possible. Popular choices include the native \LaTeX{} packages
  Ti\textit{k}Z/PGF\footnote{\url{http://www.ctan.org/pkg/pgf}} or
  Pgfplots\footnote{\url{http://www.ctan.org/pkg/pgfplots}} as well as PDF and
  SVG files.

\item For authors from the same institution there should be one common footnote
  and the email addresses in the footnote text should be written as
  \texttt{\{authorA,authorB\}@uofi.edu}.

%
% TODO: adapt numbering description, e.g. 'consecutive numbering'.
%
\item Examples, remarks, algorithms, etc., should be numbered in the same way
  and format as theorems, namely as ``Example X.Y'', where X is the section
  number and Y is the subsection number (if applicable). To generate the
  theorem-like environments ``Example'', ``Remark'', and ``Algorithm'' you
  should add the commands
\begin{minted}{latex}
\newtheorem{remark}[theorem]{Remark}
\newtheorem{example}[theorem]{Example}
\newtheorem{algorithm}[theorem]{Algorithm}
\end{minted}
to the preamble of your \LaTeX-file.
%
\item The text of examples and remarks should be typeset upright (not in
  italics as in definitions, theorems, lemmas, and corollaries).  This can be
  achieved by
\begin{minted}{latex}
\begin{remark}\upshape
 Text of remark.
\end{remark}
\end{minted}
%
\item  At the end of a proof, there should be a gap between the last word and
  the ``endproof'' symbol (Halmos box). This is achieved as follows:
\begin{minted}{latex}
\begin{proof}Text of the proof. $\qquad$ \end{proof}
\end{minted}
If the proof ends with a displayed equation you should use the following:
\begin{minted}{latex}
\emph{Proof.} Text of the proof.
\[
 a^2+b^2=c^2. \qquad\endproof
\]
\end{minted}

\item Equations that are not referenced should not be numbered.

\item Tables should be typeset with the booktabs
  package\footnote{\url{http://www.ctan.org/tex-archive/macros/latex/contrib/booktabs/}}
  and should have as few lines as possible and no vertical lines.  For example,
  instead of
\begin{center}
  \begin{tabular}{|c|c|c|}\hline
  $C1$ & $C2$ & $C3$ \\ \hline
  $a_1$ & $a_2$ & $a_3$ \\ \hline
  $b_1$ & $b_2$ & $b_3$ \\ \hline
  \end{tabular}
\end{center}
the form
\begin{center}
  \begin{tabular}{ccc}\toprule
  $C1$ & $C2$ & $C3$ \\ \midrule
  $a_1$ & $a_2$ & $a_3$ \\
  $b_1$ & $b_2$ & $b_3$\\
  \bottomrule
  \end{tabular}
\end{center}
should be used.
\item Table captions should be \emph{above} the table (unlike figure captions
  which should be below the figure). The text of all captions (table or figure)
  should end with a period~`.'.

\item Appendices should be at the end of the paper, immediately before the
  bibliography.

\item The symbols used for real and complex numbers are \verb|\mathbb{R}|,
  \verb|\mathbb{C}|, respectively.

\item For matrices, square brackets should be used
(either through \verb|\left[...\right]| or \verb|\bmatrix|).

\item Sections should be referred to by ``Section~X.Y'' and not ``\S~X.Y''.

\item The abbreviations ``et al.'', ``i.e.'', and ``e.g.'', in the text should
  always be upright. In English, there is always a comma after ``i.e.'' and
  ``e.g.''. A parenthetical remark in the middle of the sentence (e.g., this
  one) is in parenthesis, while at the end of the sentence follows a semicolon;
  e.g., this one.
\end{itemize}


\subsection{References}
To facilitate the editing process, authors are especially urged to carefully
prepare the references of their manuscripts. It is the authors' responsibility
to provide complete details such as editors, publisher, city of publication,
page numbers, department and institution, as well as correct abbreviations of
names of serials. All of this information can be found, for example, on
MathSciNet\footnote{\url{http://www.ams.org/mathscinet/}}; the tool
bibtex-mathscinet\footnote{\url{https://github.com/nschloe/bibtex-mathscinet}}
can be used to automatically adapt existing BibTeX files.

\emph{Manuscripts may be returned to authors if the manuscripts and the
references are not properly prepared.}

The preferred way for formatting the references is to use BibTeX with the SIAM
bibliography style:
\begin{minted}{latex}
\bibliographystyle{siam}
\bibliography{YourBibFile}
\end{minted}

Further instructions concerning the references:

\begin{itemize}
\item In the text references should be cited using the command
  \verb|\cite{keylist}|, where \verb|keylist| contains a list of keys (i.e.,
  the names of your bibitems), separated by commas. Example:
  \verb|\cite{ArbGol88,ArbGol95}|.

\item Citations should be either sorted according to numbers, that is [3,5,12]
  instead of [12,3,5], or by publication dates. The order should be consistent
  throughout the paper.

\item Only references cited in the text should be included in the bibliography.

\item Citations in the abstract should be avoided. If this is not possible,
  then the authors' names and the publication details should be given. For
  example,
\begin{quote}
  Golub and Kahan [SIAM J. Numer. Anal., 2 (1965), pp.~205--224] show that
  [\dots].
\end{quote}

\item Some journals have ``paper numbers'' instead of conventional page
  numbers.  An example is the Journal of Fluids Engineering. Papers in such
  journals should be cited using the paper number and the number of pages,
  e.g., the BibTeX \texttt{PAGES} field should be of the form \texttt{051202
  (10 pages)}.

\end{itemize}

\section{Submission}
To submit a manuscript, send a PDF file containing the whole manuscript by
email to \href{mailto:etna@math.kent.edu}{\nolinkurl{etna@math.kent.edu}}.

\end{document}
